\chapter{Endormissement de tache sous priorité Fixe}
\minitoc
\section{modéle de tâches}
Le modèle utilisé ici est le modèle de tâche périodique de Liu et
Layland défini au chapitre 1.\\ Soit $\taskset =
\{\task{1},\task{2},\cdots,\task{n}\}$ un ensemble de tache, chaque
tache $\task{i}$ est caracterisé par $\task{i} =
(\charge{i},\deadline{i},\period{i})$ et :
\begin{itemize}
\item $\task{i}$ est periodique.
\item $\charge{i}$ est le pire temps d'execution de la tache $i$.
\item $\deadline{i}$ est l'écheance relative de la tache $i$.
\item $\period{i}$ est la periode relative de la tache $i$.
\item l'ensemble de tâches $\taskset{}$ est ordonnançable avec
  l'algorithme Deadline Monotonic.
\end{itemize}
\section{Le cas monoprocesseur}
Dans cette section nous allons inserer une tâche d'endormissement
$\task{\sleep} = \{ \charge{\sleep}, \deadline{\sleep},
\period{\sleep}\}$ avec $\min \charge{\sleep} \leq \charge{\sleep}
\geq \max \charge{sleep}$ avec $\wcet{sleepMax} = (1-\util{}) \times
\period{H}$ et $\task{sleep} \cup \taskset{}$ est ordonnançable avec
Deadline Monotonic.


Pour cela nous presentons l'algorithme d'insertion de tache
d'endormissement $\task{\textsc{sleep}}$ dans un taskset $\taskset{}$.



\begin{algorithm}
\caption{Insertion Taches Endormissement Dans Un Mono-processeur}
\label{dmup}
\begin{algorithmic}
\State TaskSet $\taskset{}$, Temps Minimum d'execution $\wcetf_{SleepMin}$, Temps d'execution Maximum $\wcetf_{SleepMax}$, Pas de decrementation $\Delta \wcetf_{}$
\State Tache $\task{Sleep}$
\State $\wcetf_{Sleep}  \longleftarrow \wcetf_{SleepMax}$ 
\State $\deadline{Sleep}	\longleftarrow \period{H}$ 
\State $\period{sleep} \longleftarrow \period{H}$ 
\While {$\taskset{} \cup tache(\wcetf_{Sleep},\deadline{Sleep},\period{H})$ est non  ordonnançable et $\wcetf_{Sleep} \geq \wcetf_{SleepMin}$}
\State $\wcetf_{Sleep} \longleftarrow \wcetf_{Sleep} - \Delta \wcetf_{}$ 
\EndWhile 
\If {$\taskset{} \cup \task{sleep}$ est non ordonnançable \textbf{or} $\wcetf_{Sleep} \leq \wcetf_{SleepMin}$ \textbf{or} $!test(\task{sleep})$}
\State $\emptyset$
\EndIf
\State $\task{Sleep} \leftarrow Tache(\wcetf_{Sleep},\deadline{Sleep},\period{sleep})$
\end{algorithmic}
\end{algorithm}

Nous illustrons notre algorithme avec un
exemple d'application. Le tableau \ref{tab:exempledmup} et la figure
\ref{fig:exempledmupapr} represente un ensemble de tâches $\taskset{}$
ordonnançable par Deadline Monotonic où on a inseré une tache
d'endormissement $\task{\textsc{sleep}}$:

\begin{table}[!h]
\begin{center}
\begin{tabular}{|c|c|c|c|}
 \hline$\task{i}$ & $\charge{i}$ & $\deadline{i}$ & $\period{i}$ \\ 
 \hline 1 & 1 & 10 & 10 \\ 
 \hline 2 & 2 & 15 & 15 \\ 
 \hline 
 \end{tabular}
\end{center}
\caption{Ensemble de t\^aches periodiques} \label{tab:exempledmup}
\end{table}

%\begin{figure}[h]
%\begin{center}
%\begin{RTGrid}[height=4cm,width=12cm,labelsize=8pt,numbersize=6]{3}{31}
%\multido{\n=0+10}{3}{
%\TaskArrDead{1}{\n}{10}}
%\TaskArrDead{1}{20}{10}

%\TaskExecution{1}{3}{4}
%\TaskExecution{1}{13}{14}
%\TaskExecution{1}{23}{24}

%\multido{\n=0+15}{2}{
%\TaskArrDead{2}{\n}{15}}
%\TaskArrDead{2}{15}{15}
%\TaskExecution{2}{4}{5}
%\TaskExecution{2}{8}{9}
%\TaskExecution{2}{18}{20}

%\multido{\n=0+5}{6}{
%\TaskArrDead{3}{\n}{5}}
%\TaskArrDead{3}{25}{5}
%\TaskExecution{3}{0}{3}
%\TaskExecution{3}{5}{8}
%\TaskExecution{3}{10}{13}
%\TaskExecution{3}{15}{18}
%\TaskExecution{3}{20}{23}
%\TaskExecution{3}{25}{28}

%\end{RTGrid}
%\caption{Insertion de tache dans un monoprocesseur} \label{fig:exempledmup}
%\end{center}
%\end{figure}

\section{Le cas multiprocesseur}
Dans cette section nous allons inserer un ensemble de tâches
d'endormissement
$\{\task{\sleep}^1,\task{\sleep}^2,...,\task{\sleep}^m\}$ tel que
$\task{\sleep}^i = \{ \charge{\sleep}^i, \deadline{\sleep}^i,
\period{\sleep}^i\}$ dans m processeurs $\proc{} =
\{\proc{1},\proc{2},...,\proc{m}\}$.  \indent Pour cela nous
presentons nous presentons deux strategie d'insertion :

\begin{description}
\item{Insertion Locale :} Chaque processeur$_i$ à sa propre tache
  d'endormissement $\task{\sleep}^i$
\item{Insertion Globale :} Tous les processeur ont une même tache
  d'endormissement
  $\task{\sleep}=\task{\sleep}^1=\task{\sleep}^2=...=\task{\sleep}^m$
\end{description}
Les deux algorithmes representent l'inseretion locale (Resp. globale)
d'un ensemble de tâches d'endormissement dans un ensemble de
processeur.


\begin{algorithm}
\caption{Algorithme d'insertion de Tache d'endormissement en locale dans un multiprocesseur}
\label{dmmpl}
\begin{algorithmic}
\State Ensemble de processeur $\core{}$, Temps Minimum d'exécution $\wcetf_{SleepMin}$
\State Ensemble de Tache d'endormissement $\taskset{Sleep}$
\For{$ \proc{i} \in \core{}$}
\State $\task{sleep}^i$ = tache\_endormissement\_monoprocesseur($\wcetf_{SleepMin}$) 
\State $\taskset{Sleep}^i =\taskset{Sleep}^i \cup \task{sleep}^i$
\EndFor
\end{algorithmic}
\end{algorithm}

\begin{algorithm}
\caption{Algorithme d'insertion de Tache d'endormissement en globale dans un multiprocesseur}
\label{dmmpg}
\begin{algorithmic}
\State Ensemble de processeur $\core{}$, Temps Minimum d'exécution $\wcetf_{SleepMin}$
\State Ensemble de Tache d'endormissement $\taskset{Sleep}$
\State $\taskset{sleep} \leftarrow \emptyset$
\State $\task{sleep} \leftarrow \emptyset$
\State $\thrf \leftarrow PeriodeHarmonic(\taskset{Sleep})$
\For{$ \proc{i} \in \core{}$}
\State $\task{sleep}^i$ = tache\_endormissement\_monoprocesseur($\wcetf_{SleepMin},\thrf$) 
\State $\task{sleep} \leftarrow \task{sleep} \cup task{sleep}^i$
\EndFor
\For{$ \proc{i} \in \core{}$}
\State $\taskset{Sleep} \leftarrow \taskset{Sleep} \cup Min_{i =1..m}(\task{Sleep}^i)$
\EndFor
\end{algorithmic}
\end{algorithm}


Nous illustrons notre algorithme avec un exemple d'application. Le
tableau \ref{tab:exempledmmp} et les figures \ref{fig:exempledmmpl} et
\ref{fig:exempledmmpg} representent un ensemble de tâches $\taskset{}$
ordonnançable par Deadline Monotonic dans un multiprocesseur à 2
processeur en partitionnement FirstFit où on a inseré deux taches
d'endormissements $\task{\sleep}$ en locale et en globale.

\begin{figure}[!h]
\begin{center}
\begin{RTGrid}[width=10cm,nosymbols=1]{5}{31}
\RowLabel{3}{$\task{Sleep}$}
\TaskExecution[color=green]{3}{0}{4}
\TaskExecution[color=green]{3}{10}{14}
\TaskExecution[color=green]{3}{20}{24}
\RowLabel{1}{$\task{1}$}
\TaskExecution{1}{4}{9}
\TaskExecution{1}{14}{19}
\TaskExecution{1}{24}{29}
\RowLabel{2}{$\task{3}$}
\TaskExecution{2}{9}{10}
\TaskExecution{2}{19}{20}
\TaskExecution{2}{29}{30}
\RowLabel{4}{$\task{2}$}
\RowLabel{5}{$\task{Sleep}$}
\TaskExecution[color=green]{5}{0}{8}
\TaskExecution{4}{8}{15}
\TaskExecution[color=green]{5}{21}{28}
\TaskExecution{4}{28}{25}
\TaskNArrDead{1}{0}{10}{10}{3}
\TaskArrival{1}{30}
\TaskNArrDead{2}{0}{22}{24}{1}
\TaskArrival{2}{24}
\TaskNArrDead{3}{0}{10}{10}{3}
\TaskArrival{3}{30}
\TaskNArrDead{4}{0}{15}{21}{1}
\TaskArrival{4}{21}
\TaskNArrDead{5}{0}{15}{15}{2}
\TaskArrival{5}{30}
\end{RTGrid}
\end{center}
\caption{Insertion locale de tâches d'endormissements dans un multiprocesseur} \label{fig:dmmpl}
\end{figure}


\begin{figure}[!h]
\begin{center}
\begin{RTGrid}[width=10cm,nosymbols=1]{5}{31}

\RowLabel{3}{$\task{Sleep}$}
\RowLabel{1}{$\task{1}$}
\RowLabel{2}{$\task{3}$}
\RowLabel{4}{$\task{2}$}
\RowLabel{5}{$\task{Sleep}$}
\TaskNArrDead{1}{0}{10}{10}{3}
\TaskArrival{1}{30}
\TaskNArrDead{2}{0}{22}{24}{1}
\TaskArrival{2}{24}
\TaskNArrDead{3}{0}{5}{5}{6}
\TaskArrival{3}{30}
\TaskNArrDead{4}{0}{15}{21}{1}
\TaskArrival{4}{21}
\TaskNArrDead{5}{0}{5}{5}{6}
\TaskArrival{5}{30}



\TaskExecution[color=green]{3}{0}{4}
\TaskExecution[color=green]{3}{10}{14}
\TaskExecution[color=green]{3}{20}{24}


\TaskExecution{1}{4}{9}
\TaskExecution{1}{14}{19}
\TaskExecution{1}{24}{29}


\TaskExecution{2}{9}{10}
\TaskExecution{2}{19}{20}
\TaskExecution{2}{29}{30}

\TaskExecution[color=green]{5}{0}{4}
\TaskExecution[color=green]{5}{10}{14}
\TaskExecution[color=green]{5}{20}{24}

\TaskExecution{4}{4}{11}
\TaskExecution{4}{24}{30}

\end{RTGrid}
\end{center}
\caption{Insertion globale de tâches d'endormissements dans un multiprocesseur} \label{fig:dmmpg}
\end{figure}
