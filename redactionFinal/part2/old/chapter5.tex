\chapter{Expérimentations}
\section{La génération des taches}
\subsection{Cas monoprocesseur}
\subsubsection{Algorithmes UUnifast}
L’algorithme UUnifast \cite{} est un algorithme mis au point pour la
génération de taux d’utilisation sur monoprocesseur. Il génère une
distribution uniforme de n taux d’utilisation non biaisés à partir du
nombre de tâches n de l’ensemble et du taux d’utilisation processeur
total souhaité U.  UUnifast est un algorithme efficace de complexité
O(n). Nous rappelons qu’un ensemble au taux d’utilisation supérieur à
1 est trivialement non ordonnançable puisque l’utilisation processeur
dépasse alors le temps maximal disponible.
\subsubsection{Génération des périodes}
Lors de la génération de tâches, le choix des périodes est un élément
sensible pour les tests d’ordonnançabilité. En effet, certains de ces
tests basent leur analyse sur un intervalle de faisabilité.  La
longueur de cet intervalle dépend du plus petit commun multiple des
périodes (ppcm) appelée l’hyper-période. Si les périodes sont grandes,
premières entre elles, l’hyperpériode explose. Le défi consiste donc à
générer des périodes aléatoirement tout en limitant la taille de
l’hyper-période et c’est l’objet de la méthode de Goossens et Macq
\cite{Goossens01}.
\subsubsection{Génération des échéances}
Similairement à Goossens et Macq dans \cite{Goossens01}, nous generons
les échéances des taches generées. Nous déterminons aléatoirement
l’échéance dans un intervalle [$0.75 \times T_i$, $T_i$].  En résumé,
$Di = \lbrace T_i \times C_i) \times aleatoire(0.75 \times T_i,
T_i)\rbrace + C_i$ avec aleatoire(dmin, dmax) retourne un nombre réel
pseudo-aléatoire uniformément distribué sur l’intervalle [dmin, dmax]
et où la fonction arrondi(x) retourne l’entier le plus proche de x.
\subsection{Cas multiprocesseur}
\subsubsection{Algorithme UUnifast-Discard}
La méthode UUnifast présentée dans le cas monoprocesseur n’est pas
utilisée en contexte multiprocesseur, lorsque le taux d’utilisation du
processeur U peut dépasser 1. En effet, lorsque que le taux
d’utilisation total dépasse 1, UUnifast présente le risque de générer
des taux d’utilisation par tâche supérieurs à 1. Tâche qu’il n’est
alors possible d’ordonnancer sur aucun processeur.  Pour y remédier,
Davis et Burns \cite{DB11} ont proposé une extension appelée
UUnifast-Discard. Elle consiste simplement à employer UUnifast avec U
supérieur à 1 et à rejeter les ensembles pour lesquelles au moins un
taux d’utilisation par tâche est supérieur à 1. Son implantation est
simple mais cette méthode a l’inconvénient d’être particulièrement
inefficace lorsque U approche $\frac{n}{2}$ \cite{Emb10}.
\section{La simulation}
\section{Discussions}
