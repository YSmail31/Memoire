\chapter*{Conclusion \& perspectives}
\addchaptertocentry{Conclusion \& Perspactives}
\section*{Conclusions}
In this thesis, we investigate the choices for parallelization and
allocation of a set of real-time tasks to different types of multicore
platforms. As parallelization consists of decomposing a task into
several sub-task that cooperate to solve the same problem, an overhead
is assigned to parallel execution. These overheads are mainly due to
the sub-task creation, termination and synchronization.

The parallel real-time models proposed recently in the literature do
not take into account several realistic parameters such as the
parallelization costs, the static/dynamic parallel grain size
definition and does not take into account heterogeneous cores in
computing platforms such as ARM bigLITTLE.  Moreover, only few works
consider the parallelization and reducing the energy
consumption. Especially because of the slow developement of battery
technologies against the increasing energy demand in nowadays systems
such as CPS. Hence, In this thesis we had chosen to focus on the
limits of the models proposed in the litterature to cope with the
nowadays computing needs.

Particularly, in chapter 4 we addressed the problem with a simple
moldable task model that does not take into account parallelization
costs, but that solves already the problem of the representation of
difference parallel grain size for the same task. We proposed two
solutions for the problem: an exact solution by modeling the problem
as a linear problem and it was solved using lp\_solve solver, and an
heuristic that by experimentation shown that the obtained results were
very close to optimal ones.  

In the 5$^{th}$ chapter we first build a realistic timing and energy
models for a parallel execution on heterogeneous platforms. Based on
different parallelization techniques such those proposed in Cilk or
OpenMP. We proposed a task model that takes into account the
parallelization costs. The problem consists in allocating a set of
task modeled by a realistic parallel model to a set of heterogeneous
cores with the ability to calibrate the core frequency and state to
reduce the energy consumption. The problem in hand now is very hard
and was solved by expressing it as a Mixed Integer Non Linear problem
and only small-size problem could be solved using Knitro
Solver. Hence, we proposed a heuristic that allow to have
quasi-optimal solutions in a very short time.

As the two proposed models where for moldable tasks, and are not
expressive enough to cope with the increasing complexity of CPS
applications and their dynamic behavior, we proposed in Chapter 6 a
methodology to present parallel tasks with di-graphs. The proposed
model was very expressive against the models proposed in the
literrature. We addressed also the problem of allocating such tasks to
a set of identical cores with the ability of setting the core
frequency and state. The problem here is very hard, and finding an
optimal solution for a very small problems is very computational. Thus
we proposed several techniques, methods, and mathematical proofs to
reduce the complexity of the problem. The proposed methods where used
to propose a heuristic that solves the problem in a reasonable
time. The results obtained by testing a very large set of expirements
have shown that our model and scheduling heuristics are effective
against less expressive models and that it can be used to reduce the
energy consumption.
\section*{Limitations \& perspectives}
\subsection*{Shared ressources}
In this thesis we addressed the problem of parallelization under the
assumption that parallel threads communication waiting time is
bounded, and that it is included in the worst case execution time
analysis. In the last chapter, the communication between tasks are
expressed by the precedence order between different instances of a
task. However, this parameter is bounded by the worst case which is
very pessimistic. It may be possible to apply shared resources
techniques proposed in the litterature of real-time scheduling to
reproduce more realistic.
\subsection*{Global scheduling}
All our contributions reported in this thesis concern only partitioned
scheduling. The basic idea is that for heterogeneous computing
platforms, job-level migration is not allowed especially at job
migration because the timing analysis of the preempted thread at the
preemption point can not be done. However, it may be possible to allow
migration of the same thread for the cores of same group.

\subsection*{Dynamic voltage and frequency scaling}
When a parallel thread ended its execution shorter than the worst case
execution time, a slack time is engendred by these early end. Thus, it
may be convenient to recalibrate dynamically the core frequency to
allow to have less slack time, and may be save energy.

%\subsection*{Open MP design codes}
%\subsection*{WCET estimation for parallel tasks}

\chapter*{Personal publications}
\addchaptertocentry{Personal publications}
\begin{enumerate}
\item H.E. Zahaf, A.E. Benyamina, R. Olejnik, Giuseppe Lipari, Pierre Boulet
  ``Modeling parallel tasks with di-graphs'', RTNS2016 \cite{me1}

\item H.E. Zahaf, A.E. Benyamina, R. Olejnik, Giuseppe Lipari
  ``Energy-efficient partitionning for periodic soft Real-Time Tasks
  on single-ISA heterogeneous Architectures'', under final revision to
  Journal Of System Architecture \cite{me2}

\item H.E. Zahaf, R. Olejnik, G. Lipari, A.E Benyamina , "Modelling
  the Energy Consumption of Soft Real-Time Tasks on Heterogeneous
  Computing Architectures", EEHCO'2016: Energy Efficiency with
  Heterogenous Computing, Prague, January 15-16, 2016\cite{me3}

\item H.E. Zahaf, R. Olejnik, G. Lipari, A.E Benyamina , "Energy-aware
  parallel tasks scheduling on multicore architectures", ACACES
  Workshop'2015: Advanced Computer Architecture and Compilation for
  High-Performance and Embedded Systems, Fiuggi , July 12-18, 2015
  \cite{me4}
  
\item H.E. Zahaf, R. Olejnik, G. Lipari, A.E Benyamina , "Energy-aware
  moldable real-time task scheduling on uniform architectures",
  EDiS'2015: Embedded and Distributed Systems, Oran, November 15-16,
  2015 \cite{me5}
    
\item H.E. Zahaf, A.E. Benyamina, R. Olejnik "Intensive Real-Time Task
  Scheduling on Uniform Multiprocessors", Models, Optimization, and
  Mathematical analysis (MOMA), Special Issue
  IWMCS'2014,ISSN2253-0665(2014), Vol 2, Issue 1, Pages 3-13,\cite{me6}
  
\item H.E. Zahaf, A.E. Benyamina, R. Olejnik "Intensive Real-Time Task
  Scheduling on Uniform Multiprocessors",The 2 nd international
  Workshop on Mathamatics and Computer Science IWMCS'2014, december
  1-3, 2014, Tiaret, Algeria.\cite{me7}
    
\item H.E. Zahaf, A.E. Benyamina, R. Olejnik "Energy-Aware Work Load
  for Real-Time MapReduce Environment",The 1st doctorial days of
  LAPECI, Oran, September 28-29, 2014, Oran, Algeria.\cite{me8}

\item H.E. Zahaf, A.E. Benyamina, R. Olejnik "Smart cities, Scenarios
  and Applications (Poster)", National Exhibition of Valuation Research
  Result Programmes, April 8-9, 2014, Oran, Algeria.\cite{me9}
\end{enumerate}
